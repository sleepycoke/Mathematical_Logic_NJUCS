\documentclass{article}

\usepackage{fontspec, xunicode, xltxtra}
\usepackage{amsmath, amsfonts, amssymb}
\usepackage[a4paper,top=30mm,bottom=30mm,left=30mm,right=30mm]{geometry}
\usepackage{amsthm}
\usepackage[shortlabels]{enumitem}

\begin{document}
\title{A Proof of Dual Replacement Lemma}
\author{DING Chao}
\maketitle

Let $\varphi$ be a formula in basic modal and we denote $\varphi'$ as the formula obtained by replacing a ``$\lozenge$'' inside $\varphi$ with ``$\neg \square \neg$''. We claim that 
	$$\vdash_K \varphi \leftrightarrow \varphi'. $$
\begin{proof}
	
We shall prove this by structural induction on $\varphi$. 
\begin{enumerate}[{Case} 1: ]
	\item $\varphi = \lozenge \psi$ and we are replacing the first $\lozenge$, then $\varphi' = \neg \square \neg \varphi$.
	\begin{enumerate}[(1)]
		\item $\lozenge p \leftrightarrow \neg \square \neg p$ \hfill DUAL
		\item $\lozenge \varphi \leftrightarrow \neg \square \neg \varphi$ \hfill US:(1)
	\end{enumerate} 

	\item $\varphi = \neg \psi$, $\varphi' = \neg \psi'$.	 We informally claim that the definition of $\psi'$ is clear: For this case, if $\varphi'$ replaces the $i$th $\lozenge$, $\psi'$ does the $i$th. 
	\begin{enumerate}[(1)]
		\item $(p \leftrightarrow q) \rightarrow (\neg p \leftrightarrow \neg q)$ \hfill TAUT
		\item $(\psi \leftrightarrow \psi') \leftrightarrow (\neg \psi \leftrightarrow \neg \psi')$ \hfill US:(1)
		\item $\psi \leftrightarrow \psi'$ \hfill I.H.
		\item $\neg \psi \leftrightarrow \neg \psi'$ \hfill MP(2)(3)
	\end{enumerate}
	\item $\varphi = \psi * \rho$, $\varphi' = \psi' * \rho$, where $* \in \{\wedge, \vee, \rightarrow, \leftrightarrow\}$. 
	\begin{enumerate}[(1)]
			\item $(p \leftrightarrow q) \rightarrow ((p * r )\leftrightarrow (q * r) )$ \hfill TAUT
		\item $(\psi \leftrightarrow \psi') \rightarrow ((\psi * \rho) \leftrightarrow (\psi' * \rho))$ \hfill US:(1)
		\item $\psi \leftrightarrow \psi'$ \hfill I.H.
		\item $(\psi * \rho) \leftrightarrow (\psi' * \rho) $ \hfill MP(2)(3)
	\end{enumerate}
	\item $\varphi = \psi * \rho$, $\varphi' = \psi * \rho'$. A similar proof as above holds. 
	\item $\varphi = \square \psi$, $\varphi' = \square \psi'$. 
	\begin{enumerate}[(1)]
	\item $\psi \rightarrow \psi'$ \hfill PL:I.H.
	\item $\square(\psi \rightarrow \psi')$ \hfill N(1)
	\item $\square(\psi \rightarrow \psi') \rightarrow (\square \psi \rightarrow \square \psi')$ \hfill US:K
	\item $\square \psi \rightarrow \square \psi'$ \hfill MP(3)(2)
	\end{enumerate}
	Analogously, $\square \psi' \rightarrow \square \psi$. Therefore $\square \psi \leftrightarrow \square \psi'$.  
	\item $\varphi = \lozenge \psi$ and we are replacing a $\lozenge$ inside $\psi$, then $\varphi' = \lozenge \psi'$.
	\begin{enumerate}[(1)]
		\item $\square \neg \psi \leftrightarrow \square \neg \psi'$ \hfill I.H.(Case 5)
		\item $\lozenge \psi \leftrightarrow \neg \square \neg \psi$ \hfill I.H.(Case 1)
		\item $\lozenge \psi' \leftrightarrow \neg \square \neg \psi'$ \hfill I.H.(Case 1)
		\item $\lozenge \psi \leftrightarrow \lozenge \psi'$ \hfill PL:(1),(2),(3)
	\end{enumerate}
	\end{enumerate}
	
	Finally we have proved $\vdash_K \varphi \leftrightarrow \varphi'$ holds for any basic modal formula $\varphi$. 
\end{proof}

%
\end{document}
